%introduction used to set the whole 

\documentclass{article}
\title{hw3}
\author{Wenhua Bao 2512664\\Yue Liu 2803140}
\usepackage{graphicx}
\usepackage{amsmath}

%TEXT
\begin{document}
\maketitle
\section{Task1}
\subsubsection{a}
Ridge coefficient is a regulizer.
To regularize the psesudo-inverse, we use it. Otherwise, we can't get a weight cofficient, when the matrix $\phi\phi^T$ is inversiable.
\subsubsection{a}
squared error loss function
$$J(\theta)=\sum_{i=1}^{m}\left(f_{\theta}\left(x^{(i)}\right)-y^{(i)}\right)^{2}$$
add ridge coefficient$\lambda$
$$J(\theta)=\frac{1}{2}\left[\sum_{i=1}^{n}\left(f_{\theta}(x)^{(i)}-y^{(i)}\right)^{2}+\sum_{j=1}^{n} \lambda\theta_{j}^{2}\right]$$
$$\begin{aligned}
J(\theta) &=\frac{1}{2}(X \theta-Y)^{\top}(X \theta-Y)+\lambda \theta^{\top} \theta \\
&=\frac{1}{2}(X \theta-Y)^{\top}(X \theta-Y)+\lambda \theta^{\top} \theta \\
&=\frac{1}{2}\left(\theta^{\top} X^{\top} X \theta-\theta^{\top} X^{\top} Y-Y^{\top} X \theta+Y^{\top}Y+\lambda \theta^{\top} \theta\right.
\end{aligned}$$
$$\begin{array}{l}
\frac{\partial J(\theta)}{\partial \theta}=X^{\top} X \theta-X^{\top} Y+\lambda \theta=0 \\
\theta=\left(X^{\top} X+\lambda I\right)^{-1} X^{\top} Y
\end{array}$$
\subsubsection{3}
degree 1:0.3843532873748282

\section{Task2}
\subsection{a}
Generative models learn prior distribution to derive posterior distribution then get classification, but discriminative models learn posterior distribution to get classification.
discriminative models: Logistical Regression
generative models: Bayesian Analysis
Discriminative models is easier to learn, because it doesn't need to learn conditional probability and directly to learn posterior.
\subsection{2b}
there's 19 samples being misclassified.
\end{document}